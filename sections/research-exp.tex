\section{Research Experience}

\begin{twocolentry}{
    July 2025 - Present
  }
  \emphbold{International Research Collaboration on Cache System using
  Flash Storage}
\end{twocolentry}
\textit{Undergraduate Researcher}

\vspace{0.10 cm}
\begin{onecolentry}
  \begin{highlights}
  \item Collaborated with \emphbold{Prof. Juncheng Yang} from
    \emphbold{Harvard University} researching on how to integrate
    machine learning into \emphbold{Flash Cache}
    to reduce unnecessary write without
    sacrificing miss ratio.
  \item Designed and implemented \emphbold{Hierarchical Cache
    Simulator} to simulate Cache Management System
    consisting of DRAM and flash device.
  \item Benchmarked commonly used algorithms such as:
    \emphbold{CLOCK, LRU, and FIFO}. We discovered that CLOCK
    would always outperform LRU while having sequential write
    operation and low write overhead.
  \item Integrated machine learning into \emphbold{CLOCK}
    algorithm as an additional decision-maker, using \emphbold{ONNX} to ensure a
    portable pipeline.
  \end{highlights}
\end{onecolentry}

\vspace{0.2 cm}

\begin{twocolentry}{
    March 2025 - October 2025
  }
  \emphbold{International Research Collaboration on the Novel Concept
  of Lazy Promotion in Cache Eviction Algorithm}
\end{twocolentry}
\textit{Undergraduate Researcher}

\vspace{0.10 cm}
\begin{onecolentry}
  \begin{highlights}
  \item Collaborated with \emphbold{Prof. Juncheng Yang} from
    \emphbold{Harvard University}
    to improve \emphbold{miss ratio} and \emphbold{efficiency} in
    cache using the novel concept of \emphbold{Lazy Promotion}.
  \item Developed experiment and processing pipeline on
    \emphbold{6300+ traces} from
    Twitter, TencentPhoto, TencentBlock, CloudPhysics,
    Wikipedia, Alibaba, and proprietary traces.
    % \item Discovered that integrating
    %   \emphbold{machine learning} into \emphbold{promotion mechanism}
    %   using \emphbold{object metadata} could only reduce
    %   \emphbold{promotion} by \emphbold{10\%} before
    %   \emphbold{miss ratio} start increasing.
  \item Implemented the concept of \emphbold{Lazy Promotion} into advanced
    algorithms such as \emphbold{ARC} and \emphbold{2Q}. Improved
    miss ratio by \emphbold{1\%} and reduced
    promotion by
    \emphbold{80\%}
  \item Discovered \emphbold{Delayed-CLOCK} which outperforms
    both \emphbold{LRU} and \emphbold{CLOCK}. Reduced miss
    ratio by \emphbold{20\%} and promotion by \emphbold{90\%}
    compared to LRU.
  \item Packaged the experiments conducted into
    \emphbold{fully reproducible artifact}
    \href{https://anonymous.4open.science/r/Lazy-Promotions-3833/README.md}{\faGithub}.
  \end{highlights}
\end{onecolentry}

\vspace{0.2 cm}

\begin{twocolentry}{
    Jan 2025 – Jun 2025
  }
  \emphbold{UChicago-Indonesia SYstem and AI Research Training}
\end{twocolentry}
\textit{Research Trainee}

\vspace{0.10 cm}
\begin{onecolentry}
  \begin{highlights}
  \item \emphbold{Top 50} students from Indonesia are selected for
    this program.
  \item Covered \emphbold{20+} papers and reproduced key experiments from
    \emphbold{OSDI, SOSP,
    FAST} conferences.
  \item Instructor: \emphbold{Prof. Haryadi Gunawi} from \emphbold{University
    of Chicago}.
  \end{highlights}
\end{onecolentry}
